\subsubsection{Bond Forward / T-Lock / J-Lock (using ref. data)}
\label{ss:BondForward_refdata}

A Forward Bond (or Bond Forward) is a contract that establishes an agreement to buy or sell (determined by
\lstinline!LongInForward!) an underlying bond at a future point in time (the {\tt ForwardMaturityDate}) at an agreed
price (the settlement {\tt Amount}).

A T-Lock is a Forward Bond with a US Treasury Bond as underlying, whereas a J-Lock is a Forward Bond with a Japanese
Government Bond as underlying. T-Locks can be specified in terms of a lock-in yield rather then a settlement
amount. The cash settlement amount is given by (bond yield at maturity - lock rate) x DV01 in this case.

Listing \ref{lst:forward_bond_refdata} shows an example for a physically settled forward bond. Listing
\ref{lst:forward_bond_refdata_tlock} shows an example for a cash settled T-Lock transaction specified by a lock-in yield.

A Forward Bond is set up using a {\tt ForwardBondData} block as shown below and the trade type is
\emph{ForwardBond}. The specific elements are

\begin{itemize}
   \item The {\tt BondData} block specifies the underlying bond, see below for more details.
   \begin{itemize}
     \item SecurityId: The underlying security identifier \\    
        Allowable values:  Typically the ISIN of the underlying bond, with the ISIN: prefix. 
     \item BondNotional: The notional of the underlying bond on which the forward is written expressed in the currency of
       the bond \\
        Allowable values:  Any positive real number.
    \item CreditRisk [Optional] Boolean flag indicating whether to show Credit Risk on the Bond product. If set to \emph{false}, the product class will be set to \emph{RatesFX} instead of \emph{Credit}, and there will be no credit sensitivities. Note that if the underlying bond reference is set up without a CreditCurveId - typically for some highly rated government bonds -  the CreditRisk flag will have no impact on the product class and no credit sensitivities will be shown even if CreditRisk is set to \emph{true}.\\
  Allowable Values: \emph{true} or \emph{false} Defaults to \emph{true} if left blank or omitted.    
   \end{itemize}
   \item SettlementData: The entity defining the terms of settlement:
   \begin{itemize}
       \item ForwardMaturityDate: The date of maturity of the forward contract. \\
         Allowable values: See \lstinline!Date! in Table \ref{tab:allow_stand_data}.
       \item ForwardSettlementDate [Optional]: Settlement date for forward bond or cash settlement payment date.  \\
         Allowable values: See \lstinline!Date! in Table \ref{tab:allow_stand_data}.
       \item Settlement [Optional]: Cash or Physical. Option, defaults to Physcial, except in case the settlement is
         defined by LockRate, in which case it defaults to Cash. \\
         Allowable values: Cash, Physical
       \item Amount [Optional]: The settlement amount (also called strike) transferred at forward maturity in return for
         the bond (physical delivery) or a cash amount equal to the dirty price of the bond (cash settlement). This is
         transferred from the party that is long to the party that is short (determined by \lstinline!LongInForward!)
         and cannot be a negative amount. It is assumed to be in the same currency as the underlying bond. Exactly one
         of the fields Amount, LockRate must be given. \\
         Allowable values: Any non-negative real number. 
       \item LockRate [Optional]: The payoff is given by (yield at forward maturity - LockRate) x DV01 (LongInForward =
         true). Exactly one of the fields Amount, LockRate must be given. In case the LockRate is given, the Settlement
         must be set to Cash. If Settlement is not given, it defaults to Cash in this case. \\
         Allowable values: Any non-negative real number. The LockRate is expressed in decimal form, eg 0.05 is a rate of 5\%
       \item dv01 [Optional]: When the LockRate is given, it is possible to implement a contractual DV01 instead of deriving it from the bond price. \\
         Allowable values: Any positive real number. E.G If the dPdY is given then dv01=10000*dPdY/N.
       \item LockRateDayCounter [Optional]: The day counter w.r.t. which the lock rate is expressed. Optional, defaults to A360. \\
         Allowable values: see table \ref{tab:daycount}
       \item SettlementDirty [Optional]: A flag that determines whether the settlement amount {({\tt Amount})} reflects
         a clean (\emph{false}) or dirty (\emph{true}) price. In either case, the dirty amount is actually paid on the
         forward maturity date, i.e. if SettlementDirty = \emph{false}, the (forward) accruals are computed internally
         and added to the given amount to get the actual settlement amount. Optional, defaults to true. \\
         Allowable values: \emph{true}, \emph{false}
   \end{itemize}
   \item PremiumData: The entity defining the terms of a potential premium payment. This node is optional. If left out it is assumed that no premium is paid.
   \begin{itemize}
       \item Date: The date when a premium is paid. \\
       Allowable values: See \lstinline!Date! in Table \ref{tab:allow_stand_data}.       
       \item Amount: The amount transferred as a premium. This is transferred from the party that is long to the party
         that is short (determined by \lstinline!LongInForward!) and cannot be a negative amount. It is assumed to be in
         the same currency as the underlying bond.\\
         Allowable values: Any non-negative real number.
   \end{itemize}
   \item LongInForward: A flag that determines whether the forward contract is entered in long (\emph{true}) or short
     (\emph{false}) position. \\
       Allowable values: \emph{true}, \emph{false}     
\end{itemize}
 
\begin{listing}[H]
\begin{minted}[fontsize=\small]{xml}
   <ForwardBondData>
     <BondData>
       <SecurityId>ISIN:XS1234567890</SecurityId>
       <BondNotional>100000</BondNotional>
     <BondData>
     <SettlementData>
       <ForwardMaturityDate>20160808</ForwardMaturityDate>
       <Settlement>Physcial</Settlement>
       <ForwardSettlementDate>20160810</ForwardSettlementDate>
       <Amount>1000000.00</Amount>
       <SettlementDirty>true</SettlementDirty>
     </SettlementData>
     <PremiumData>
       <Amount>1000.00</Amount>
       <Date>20160808</Date>
     </PremiumData>
     <LongInForward>true</LongInForward>
   </ForwardBondData>
\end{minted}
\caption{Forward Bond Data}
\label{lst:forward_bond_refdata}
\end{listing}

\begin{listing}[H]
   \begin{minted}[fontsize=\small]{xml}
   <ForwardBondData>
     <BondData>
       <SecurityId>ISIN:XS1234567890</SecurityId>
       <BondNotional>100000</BondNotional>
     </BondData>
     <SettlementData>
       <ForwardMaturityDate>20160808</ForwardMaturityDate>
       <ForwardSettlementDate>20160810</ForwardSettlementDate>
       <LockRate>0.02365</LockRate>
     </SettlementData>
     <LongInForward>true</LongInForward>
   </ForwardBondData>
\end{minted}
\caption{Forward Bond Date (T-Lock)}
\label{lst:forward_bond_refdata_tlock}
\end{listing}

\begin{listing}[H]
	\begin{minted}[fontsize=\small]{xml}
		<ForwardBondData>
		<BondData>
		  <SecurityId>ISIN:XS1234567890</SecurityId>
		  <BondNotional>100000</BondNotional>
		</BondData>
		<SettlementData>
		  <ForwardMaturityDate>20160808</ForwardMaturityDate>
		  <ForwardSettlementDate>20160810</ForwardSettlementDate>
		  <LockRate>0.02365</LockRate>
		  <dv01>0.8</dv01>
		</SettlementData>
		<LongInForward>true</LongInForward>
		</ForwardBondData>
	\end{minted}
	\caption{Forward Bond Date (T-Lock) with DV01}
	\label{lst:forward_bond_refdata_tlock_dv01}
\end{listing}
