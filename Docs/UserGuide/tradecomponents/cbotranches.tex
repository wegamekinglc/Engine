\subsubsection{CBO Tranches}
\label{ss:cbotranches} 

This trade component node is used in a CBO trade as explained in \ref{ss:CBOData}.
An example structure of the \lstinline!CBOTranches! trade component node is shown in Listing \ref{lst:cbotranches}.

\begin{listing}[H]
%\hrule\medskip
\begin{minted}[fontsize=\footnotesize]{xml}
<CBOTranches>
  <Tranche>
    <Name>JuniorNote</Name>
    <ICRatio>0.0</ICRatio>
    <OCRatio>0.0</OCRatio>
    <Notional>4000000.00</Notional>
    <FixedLegData>
      <Rates>
        <Rate>0.03</Rate>
      </Rates>
    </FixedLegData>
  </Tranche>
  ...
</CBOTranches>

\end{minted}
\caption{CBO Tranches}
\label{lst:cbotranches}
\end{listing}

The meanings of the elements of the {\tt CBO tranches} node follow below:

\begin{itemize}
\item Tranche: Multiple tranches are allowed and are indicated by the tranche node within the embracing CBOTranches node. 

\item Name: This string is the name of the tranche, possibly reflecting the position in the capital structure.  

\item ICRatio: The interest coverage ratio is a number, defined as BasketInterest over TrancheInterest (incl. all senior tranches).

\item OCRatio: The overcollateralisation ratio is a number, defined as BasketNotional over TrancheNotional (incl. all senior tranches).

\item Notional: The face amount of the tranche.

\end{itemize}

Depending on the tranche, one can specify a floating or fixed return via the nodes:

\begin{itemize}
\item FixedLegData, which is outlined in section \ref{ss:fixedleg_data}.

\item FloatingLegData, which is outlined in section \ref{ss:floatingleg_data}.

\end{itemize}