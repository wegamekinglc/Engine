Scripted trades can be used in combination with AMC analytics type. Usually a separate script will be used within this
analytics type because of the special inputs and outputs of AMC scripts as outlined below. To set up a separate script
that should be used within the AMC analytics type, the attribute purpose should be given the value \verb+AMC+, i.e. in
the script library we could have

\begin{minted}[fontsize=\footnotesize]{xml}
  <Script>
    <Name>BermudanSwaption</Name>
    <!-- default script -->
    <Script>
    ...
    </Script>
    <!-- script that will be preferred in an AMC context -->
    <Script purpose="AMC">
    ...
    </Script>
  </Script>
\end{minted}

and similar for embedded scripts. A script that in run withing the AMC analytics type gets a special input event array
\verb+_AMC_SimDates+ that contains the dates on which a conditional NPV is required as an output. This output has to be
delivered in a number array \verb+_AMC_NPV+ which has the same size. Typically one will build a new schedule from the
amc sim dates and other event dates using something like

\begin{minted}[fontsize=\footnotesize]{xml}
  <NewSchedules>
    <NewSchedule>
      <Name>ExerciseAndSimDates</Name>
      <Operation>Join</Operation>
      <Schedules>
        <Schedule>_AMC_SimDates</Schedule>
        <Schedule>ExerciseDates</Schedule>
      </Schedules>
    </NewSchedule>
  </NewSchedules>
\end{minted}

within the script node (see \ref{scriptNode}). When looping over the common event dates, the DATEINDEX() function can be
used to distinguish the different kind of events.

The AMC analytics type supports several simulation modes w.r.t. mpor grids

\begin{itemize}
\item no close-out lag
\item close-out lag with mpor mode actual date
\item close-out lag with mpor mode sticky date
\end{itemize}

In the first two cases the \verb+_AMC_SimDate+ grid will consist of all valuation and close-out dates specified in the
simulation setup. In the last case the \verb+_AMC_SimDate+ grid will consist only of the valuation date and two runs
will be performed on these dates, one using the original and another one using a time-shifted stochastic process. Since
it is not desirable that exercise decisions or barrier hit indicators are recomputed in the second run, it is possible
to define a set of variables for which values are computed in the first run and then reused in these second
run. Assignments to these variables in the second run are ignored. The definition of such variables is done in the
script node (see \ref{scriptNode}):

\begin{minted}[fontsize=\footnotesize]{xml}
  <StickyCloseOutStates>
    <StickyCloseOutState>ExerciseIndicator</StickyCloseOutState>
  </StickyCloseOutStates>
\end{minted}